\documentclass[12pt]{article}
\usepackage[utf8]{inputenc}
\usepackage{amsmath}
\usepackage{amssymb}
\usepackage{amsthm}
\usepackage{graphicx}
\usepackage{hyperref}
\usepackage{geometry}
\usepackage{enumitem}
\usepackage{listings}
\usepackage{xcolor}

\geometry{margin=1in}

\title{Sasando Acoustic Replication: Mathematical Modeling of Traditional Instrument Sound in Electric Versions}
\author{Ryan Le}
\date{\today}

\begin{document}

\maketitle

% ============================================================================
% ABSTRACT
% ============================================================================
\begin{abstract}
% TODO: Write abstract
% - Brief problem statement: Electric sasando lacks acoustic quality of traditional sasando
% - Approach: Dual Fourier series model to replicate resonator effects
% - Key method: Extract parameters from reference signals, apply to electric input
% - Results: Model successfully replicates harmonic structure using 400 harmonics
% - Significance: Foundation for preserving traditional instrument sound in electric versions
\end{abstract}

% ============================================================================
% 1. INTRODUCTION
% ============================================================================
\section{Introduction}

\subsection{Problem Statement}
% TODO: Write problem statement
% - Background on traditional sasando instrument
%   - Hand-formed, dried lontar leaf resonator
%   - Natural amplification of frequencies 98-1047 Hz
%   - Bamboo tube: 70 cm length, 8 cm diameter
% - Modern electric sasando: pickup and speaker system
% - Problem: Sacrifices acoustic quality for portability
% - Challenge: Replicate authentic acoustic sound in electric version

\subsection{Objective}
% TODO: Write objective
% - Develop mathematical model to represent acoustic characteristics
% - Replicate resonator's frequency-dependent amplification effects
% - Preserve harmonic structure and decay characteristics
% - Provide foundation for real-time sound synthesis

\subsection{Approach Overview}
% TODO: Write approach overview
% - Dual Fourier series model
% - Decompose acoustic signal into harmonic components
% - Model resonator contribution as frequency-dependent transformation
% - Extract features from recorded signals
% - Reconstruct using mathematical framework for DSP implementation

% ============================================================================
% 2. THEORY
% ============================================================================
\section{Theory}

\subsection{Notation}

\subsubsection{Time Domain Variables}
\begin{itemize}
    \item $t$: Time (seconds)
    \item $f(t)$: Signal as a function of time
    \item $E(t)$: Amplitude envelope function
    \item $A_0$: Initial amplitude of the envelope
    \item $\tau$: Decay time constant (seconds)
\end{itemize}

\subsubsection{Frequency Domain Variables}
\begin{itemize}
    \item $f_0$: Fundamental frequency (Hz)
    \item $\omega_0 = 2\pi f_0$: Fundamental angular frequency (rad/s)
    \item $n$: Harmonic index (integer: 0, 1, 2, \ldots)
    \item $nf_0$: Frequency of the $n$-th harmonic (Hz)
    \item $M_n$: Magnitude of the $n$-th harmonic from FFT
\end{itemize}

\subsubsection{Fourier Series Coefficients}
\begin{itemize}
    \item $A_n$: Amplitude coefficient for the $n$-th harmonic
    \item $\phi_n$: Phase coefficient for the $n$-th harmonic
    \item $\alpha_n$: Frequency-dependent amplitude scaling factor for the $n$-th harmonic
    \item $\theta$: Phase shift introduced by the resonator (radians)
\end{itemize}

\subsubsection{Signal Processing}
\begin{itemize}
    \item $F_s$: Sampling rate (Hz) - typically 48,000 Hz
    \item $N$: Number of samples
    \item $T$: Signal duration (seconds)
    \item $\text{FFT}[\cdot]$: Fast Fourier Transform operator
\end{itemize}

\subsubsection{Model Parameters}
\begin{itemize}
    \item $\text{num\_harmonics}$: Number of harmonics used in the model (400 in this work)
    \item $\text{SASANDO\_FREQ\_MIN} = 98$ Hz: Minimum frequency of sasando resonator range
    \item $\text{SASANDO\_FREQ\_MAX} = 1047$ Hz: Maximum frequency of sasando resonator range
\end{itemize}

\subsection{Assumptions}
\label{sec:assumptions}

The mathematical model is based on the following assumptions:

\begin{enumerate}
    \item \textbf{Harmonic Structure}: The signal can be accurately represented as a sum of harmonics (integer multiples of the fundamental frequency). This assumes that inharmonic partials are negligible or can be approximated by nearby harmonics.
    
    \item \textbf{Linear Resonator Model}: The resonator's effect is modeled as a linear transformation with frequency-dependent amplitude scaling $\alpha_n$ and constant phase shift $\theta$. This assumes the resonator behaves linearly and does not introduce significant non-linear distortion.
    
    \item \textbf{Exponential Decay}: The amplitude envelope follows an exponential decay model $A_0 e^{-t/\tau}$. This assumes that the decay is uniform across all harmonics and that the decay time constant $\tau$ is constant throughout the signal.
    
    \item \textbf{Stationary Signal}: The signal characteristics (fundamental frequency, harmonic amplitudes, phases) are assumed to be stationary over the analysis window, except for the exponential decay envelope.
    
    \item \textbf{Perfect Harmonic Alignment}: The model assumes that all frequency components are exact integer multiples of the fundamental frequency $f_0$. In reality, slight frequency variations may occur.
    
    \item \textbf{Noise-Free Extraction}: The feature extraction process assumes that noise and continuous spectral content do not significantly affect the extraction of harmonic amplitudes and phases.
    
    \item \textbf{Resonator Frequency Response}: The resonator's frequency response is approximated by a step function:
    \begin{equation}
        \alpha_n = \begin{cases}
            0.8 & \text{if } 98 \text{ Hz} \leq n \cdot f_0 \leq 1047 \text{ Hz} \\
            0.2 & \text{otherwise}
        \end{cases}
    \end{equation}
    This is a simplification of the actual complex frequency response.
    
    \item \textbf{Constant Phase Shift}: The phase shift $\theta$ introduced by the resonator is assumed to be constant across all frequencies. In reality, phase shifts may vary with frequency.
\end{enumerate}

\subsection{Sound Waves and Harmonic Decomposition}

% TODO: Write section
% - Sound waves as periodic pressure variations
% - Fourier analysis for frequency decomposition
% - Fundamental frequency ($f_0$): lowest frequency, perceived pitch
% - Harmonics: integer multiples ($nf_0$)
% - Inharmonic partials: non-integer multiples
% - Traditional sasando: vibrating strings + lontar leaf resonator
% - Resonator amplifies 98-1047 Hz range

\subsection{Problem Composition}

% TODO: Write section
% - Harmonic Components: Primary frequency content (integer multiples) - can be represented by Fourier series
% - Inharmonic Components: Non-harmonic partials, especially during attack phase - contribute to timbre but not captured by harmonic model

\subsection{Mathematical Model}

To model the interaction between the electric sasando signal and the resonator, we use a dual Fourier series representation:

\begin{equation}
\label{eq:main_model}
f(t) = \sum_{n=0}^{\infty} A_n \cos\left(n\omega_0 t + \phi_n\right) + \sum_{n=0}^{\infty} \alpha_n A_n \cos\left(n\omega_0 t + \phi_n + \theta\right)
\end{equation}

Where:
\begin{itemize}
    \item $\omega_0 = 2\pi f_0$ is the fundamental frequency in rad/s
    \item $A_n$ are the amplitude coefficients for each harmonic (extracted from the signal)
    \item $\phi_n$ are the phase coefficients for each harmonic (extracted from the signal)
    \item $\alpha_n$ is the frequency-dependent amplitude scaling factor for the $n$-th harmonic (models resonator response)
    \item $\theta$ is the phase shift introduced by the resonator (constant, typically $\pi/4$ rad)
\end{itemize}

The first sum represents the original signal decomposed into harmonics, while the second sum represents the resonator's contribution with frequency-dependent scaling $\alpha_n$ and phase shift $\theta$.

\subsubsection{Resonator Response Function}

The function $\alpha_n$ models the frequency-dependent amplification of the sasando resonator:

\begin{equation}
\label{eq:alpha_n}
\alpha_n = \begin{cases}
    0.8 & \text{if } 98 \text{ Hz} \leq n \cdot f_0 \leq 1047 \text{ Hz} \\
    0.2 & \text{otherwise}
\end{cases}
\end{equation}

This reflects the resonator's natural amplification of frequencies within the sasando's characteristic range.

\subsection{Amplitude Envelope and Decay}

Plucked string instruments exhibit exponential decay in amplitude over time. To model this, we multiply the steady-state model by an exponential envelope:

\begin{equation}
\label{eq:decay_model}
f_{\text{decay}}(t) = E(t) \cdot f(t) = A_0 e^{-t/\tau} \cdot \left[ \sum_{n=0}^{\infty} A_n \cos\left(n\omega_0 t + \phi_n\right) + \sum_{n=0}^{\infty} \alpha_n A_n \cos\left(n\omega_0 t + \phi_n + \theta\right) \right]
\end{equation}

Where:
\begin{itemize}
    \item $A_0$ is the initial amplitude (extracted from the signal's peak envelope)
    \item $\tau$ is the decay time constant (fitted from the signal's envelope using exponential regression)
\end{itemize}

The envelope is extracted using the Hilbert transform, which provides an accurate representation of the signal's amplitude modulation.

\subsection{A$_n$ Extraction Formula}

When extracting the amplitude coefficients $A_n$ from a recorded signal (which already includes resonator effects), we must account for the resonator's contribution. The FFT magnitude $M_n$ at frequency $n\omega_0$ represents the combined magnitude of both series terms.

\subsubsection{Derivation}

Starting with the sum of two cosine terms:
\begin{equation}
s(t) = A_n \cos(n\omega_0 t + \phi_n) + \alpha_n A_n \cos(n\omega_0 t + \phi_n + \theta)
\end{equation}

Using the trigonometric identity $\cos(a + b) = \cos(a)\cos(b) - \sin(a)\sin(b)$:

\begin{align}
s(t) &= A_n \cos(n\omega_0 t + \phi_n) + \alpha_n A_n [\cos(n\omega_0 t + \phi_n)\cos(\theta) - \sin(n\omega_0 t + \phi_n)\sin(\theta)] \nonumber \\
&= A_n [1 + \alpha_n \cos(\theta)] \cos(n\omega_0 t + \phi_n) - \alpha_n A_n \sin(\theta) \sin(n\omega_0 t + \phi_n)
\end{align}

This can be written as a single sinusoid $R \cos(n\omega_0 t + \phi_n + \delta)$, where the magnitude $R$ is:

\begin{align}
R &= A_n \sqrt{[1 + \alpha_n \cos(\theta)]^2 + [\alpha_n \sin(\theta)]^2} \nonumber \\
&= A_n \sqrt{1 + 2\alpha_n \cos(\theta) + \alpha_n^2(\cos^2(\theta) + \sin^2(\theta))} \nonumber \\
&= A_n \sqrt{1 + 2\alpha_n \cos(\theta) + \alpha_n^2}
\end{align}

Since $M_n$ (from FFT) represents the magnitude $R$ of the combined signal, we solve for $A_n$:

\begin{equation}
\label{eq:A_n_extraction}
A_n = \frac{M_n}{\sqrt{1 + 2\alpha_n \cos(\theta) + \alpha_n^2}}
\end{equation}

This formula correctly accounts for the phase relationship between the two cosine terms when extracting the base amplitude coefficient.

% ============================================================================
% 3. METHODS
% ============================================================================
\section{Methods}

\subsection{Data Collection}

% TODO: Write section
% - Three types of guitars: acoustic, classical, electric
% - Three musical notes: C, D, E
% - Sampling rate: 48 kHz
% - Format: M4A audio files
% - Duration: 7-10 seconds per recording
% - Total: 9 instrument-note combinations
% - Purpose: Reference signals representing acoustic characteristics

\subsection{Feature Extraction}

\subsubsection{Fundamental Frequency Estimation}
The fundamental frequency $f_0$ was estimated by identifying the peak magnitude in the FFT spectrum within the range 20-2000 Hz. This provides $\omega_0 = 2\pi f_0$ for the Fourier series model.

\subsubsection{Harmonic Detection}
For each signal, 400 harmonics were extracted ($n = 0$ to 399), ensuring comprehensive coverage of the frequency spectrum. The harmonic frequencies are calculated as $n \cdot f_0$ for each harmonic index $n$.

\subsubsection{Amplitude and Phase Extraction}
For each harmonic:
\begin{enumerate}
    \item The FFT magnitude $M_n$ at frequency $n \cdot f_0$ was extracted
    \item The phase $\phi_n$ was extracted from the FFT phase spectrum
    \item The base amplitude $A_n$ was calculated using the derived formula:
    \begin{equation}
        A_n = \frac{M_n}{\sqrt{1 + 2\alpha_n \cos(\theta) + \alpha_n^2}}
    \end{equation}
    This accounts for the resonator effect already present in the recorded signal.
\end{enumerate}

\subsubsection{Envelope Extraction}
The amplitude envelope was extracted using the Hilbert transform method, which provides an accurate representation of the signal's amplitude modulation over time. The envelope was then smoothed using a moving average filter (window size: 100 samples).

\subsubsection{Decay Parameter Fitting}
An exponential decay model $A_0 e^{-t/\tau}$ was fitted to the extracted envelope using linear regression on the logarithmic envelope values. This provides:
\begin{itemize}
    \item $A_0$: Initial amplitude (peak of the envelope)
    \item $\tau$: Decay time constant
\end{itemize}

\subsection{Model Implementation}

\subsubsection{Fourier Series Reconstruction}
The signal was reconstructed using the dual Fourier series model:
\begin{equation}
f_{\text{reconstructed}}(t) = E(t) \cdot \left[ \sum_{n=0}^{399} A_n \cos\left(n\omega_0 t + \phi_n\right) + \sum_{n=0}^{399} \alpha_n A_n \cos\left(n\omega_0 t + \phi_n + \theta\right) \right]
\end{equation}

Where:
\begin{itemize}
    \item $E(t) = A_0 e^{-t/\tau}$ is the exponential decay envelope
    \item 400 harmonics ($n = 0$ to 399) are used
    \item $\theta = \pi/4$ radians (45 degrees)
    \item $\alpha_n$ is determined by the resonator response function (Equation~\ref{eq:alpha_n})
\end{itemize}

\subsubsection{Resonator Effect}
The second series term models the resonator's contribution, with frequency-dependent scaling $\alpha_n$ and phase shift $\theta$. This dual-series approach captures both the direct signal and its interaction with the resonator.

\subsection{Evaluation Metrics}

To quantify the accuracy of the model, several metrics were computed for both time and frequency domains:

\subsubsection{Time Domain Metrics}
\begin{itemize}
    \item \textbf{Pearson Correlation Coefficient}: Measures linear correlation between original and reconstructed signals
    \item \textbf{Root Mean Squared Error (RMSE)}: Average magnitude of errors
    \item \textbf{Mean Absolute Error (MAE)}: Average absolute difference
    \item \textbf{Normalized Mean Squared Error}: MSE normalized by signal variance
\end{itemize}

\subsubsection{Frequency Domain Metrics}
\begin{itemize}
    \item \textbf{Frequency spectrum comparison}: Normalized magnitude spectra compared at each frequency
    \item \textbf{Harmonic magnitude matching}: Comparison of individual harmonic amplitudes
\end{itemize}

\subsubsection{Amplitude Ratios}
\begin{itemize}
    \item \textbf{Max Ratio}: Ratio of maximum amplitudes (Fourier/Original)
    \item \textbf{RMS Ratio}: Ratio of RMS amplitudes (Fourier/Original)
\end{itemize}

These metrics provide objective measures of how well the model replicates the original signal characteristics.

% ============================================================================
% 4. RESULTS
% ============================================================================
\section{Results}

\subsection{Harmonic Detection}

% TODO: Write section
% - 400 harmonics used for all notes
% - Fixed number ensures consistent analysis
% - Comprehensive frequency spectrum coverage

\subsection{Time Domain Matching}

% TODO: Write section
% - Waveform shape: close match, especially attack and sustain phases
% - Amplitude envelope: exponential decay captures natural decay
% - Phase alignment: preserved through $\phi_n$ coefficients

\subsection{Frequency Domain Matching}

% TODO: Write section
% - Fundamental Frequency: Accurately identified and replicated (130-330 Hz range)
% - Harmonic Magnitude: Primary harmonics well-matched, especially fundamental and lower-order
% - Higher Frequency Content: Some discrepancies due to:
%   - Continuous spectral content (noise floor) not representable by discrete harmonics
%   - Very weak harmonics below extraction threshold
%   - Inharmonic partials (non-integer multiples)

\subsection{Quantitative Metrics}

% TODO: Write section with actual numbers
% - Time Domain: Strong correlations, RMSE/MAE reflect amplitude matching, normalized MSE accounts for variance
% - Frequency Domain: Spectrum correlations show overall shape replication, individual harmonic comparisons
% - Amplitude Ratios: Max and RMS ratios near 1.0 indicate good energy matching

\subsection{Instrument-Specific Observations}

% TODO: Write section
% - Classical Guitar: Best matching, clear harmonic structure, strong fundamentals
% - Acoustic Guitar: More complex spectral content (especially note E), denser frequency distributions
% - Electric Guitar: Similar to acoustic, good fundamental matching, some higher frequency discrepancies

\subsection{Effect of Harmonic Count}

% TODO: Write section
% - 400 harmonics vs. initial 31-77: significant improvement
% - Better higher frequency representation
% - More complete spectral coverage
% - Improved amplitude matching across spectrum

% ============================================================================
% 5. DISCUSSION & CONCLUSION
% ============================================================================
\section{Discussion \& Conclusion}

\subsection{Model Performance}

\subsubsection{Strengths}
\begin{itemize}
    \item Accurate fundamental frequency identification
    \item Good primary harmonic magnitude matching
    \item Successful decay characteristics capture
    \item Effective resonator effect modeling (dual-series approach)
\end{itemize}

\subsubsection{Limitations}
\begin{itemize}
    \item Cannot represent continuous spectral content (noise floor)
    \item Limited to harmonic components (misses inharmonic partials)
    \item Simplified resonator model (step function, constant phase)
    \item Transient effects during attack may not be fully captured
\end{itemize}

\subsection{Sources of Discrepancy}

% TODO: Expand each point
% - Inherent Model Limitations: Fourier series only represents periodic harmonics; real sounds have inharmonic content
% - Feature Extraction Challenges: Weak harmonics below threshold, noise affects accuracy, $f_0$ errors propagate
% - Simplified Resonator Model: Step-function $\alpha_n$ may not match true response; constant $\theta$ may not capture frequency-dependent phase
% - Spectral Characteristics: Real FFTs show leakage/broadening; model produces ideal sharp lines; noise floor not representable

\subsection{Parameter Sensitivity}

% TODO: Write section
% - Number of Harmonics: 400 significantly improved matching; beyond 400 may have diminishing returns
% - Resonator Parameters: Choice of $\alpha_n$ function and $\theta$ affects resonator effect replication
% - Decay Parameters: Exponential model effectively captures decay; extracted $\tau$, $A_0$ provide good matches

\subsection{Application to Electric Sasando Signal Generation}

The Fourier model parameters extracted from reference signals can be directly applied to transform electric sasando input signals. The process involves:

\subsubsection{Step 1: Parameter Extraction from Reference}
Extract $A_n$, $\phi_n$, $f_0$, $\tau$, and $A_0$ from traditional sasando recordings or acoustic reference signals. These parameters define the target acoustic characteristics.

\subsubsection{Step 2: Electric Sasando Input Processing}
Take the electric sasando signal (from pickup) as input. Extract its fundamental frequency $f_0^{\text{elec}}$ and harmonic structure. This provides the base signal that needs transformation.

\subsubsection{Step 3: Signal Transformation}
The electric sasando signal $s_{\text{elec}}(t)$ is transformed using the extracted parameters:

\begin{equation}
\label{eq:electric_transformation}
s_{\text{transformed}}(t) = E(t) \cdot \left[ \sum_{n=0}^{399} A_n^{\text{ref}} \cos\left(n\omega_0^{\text{elec}} t + \phi_n^{\text{ref}}\right) + \sum_{n=0}^{399} \alpha_n A_n^{\text{ref}} \cos\left(n\omega_0^{\text{elec}} t + \phi_n^{\text{ref}} + \theta\right) \right]
\end{equation}

Where:
\begin{itemize}
    \item $A_n^{\text{ref}}$, $\phi_n^{\text{ref}}$: Amplitude and phase coefficients from reference (traditional sasando)
    \item $\omega_0^{\text{elec}} = 2\pi f_0^{\text{elec}}$: Fundamental frequency from electric sasando input
    \item $E(t) = A_0^{\text{ref}} e^{-t/\tau^{\text{ref}}}$: Decay envelope from reference
    \item $\alpha_n$, $\theta$: Resonator parameters (same for all signals)
\end{itemize}

\textbf{Key Insight}: The electric sasando provides the fundamental frequency and timing, while the reference parameters provide the harmonic structure, phases, and decay characteristics. This creates a hybrid signal that combines the electric input's pitch with the acoustic reference's timbre.

\subsubsection{Real-Time Implementation}
For real-time processing:
\begin{enumerate}
    \item Continuously extract $f_0^{\text{elec}}$ from electric sasando input
    \item Use pre-extracted $A_n^{\text{ref}}$, $\phi_n^{\text{ref}}$, $\tau^{\text{ref}}$, $A_0^{\text{ref}}$ from reference database
    \item Synthesize output signal using the transformation equation (Equation~\ref{eq:electric_transformation})
    \item Apply the decay envelope in real-time based on note onset detection
\end{enumerate}

\subsection{Applications and Future Work}

\subsubsection{Applications}
\begin{itemize}
    \item The model provides a foundation for replicating traditional sasando sound in electric versions
    \item The dual-series approach can be adapted for other resonator-based instruments
    \item The method demonstrates how mathematical modeling can preserve acoustic characteristics
\end{itemize}

\subsubsection{Future Improvements}
\begin{enumerate}
    \item \textbf{Inharmonic Modeling}: Incorporate non-harmonic partials using frequency modulation or additive synthesis
    \item \textbf{Noise Modeling}: Add a noise floor component to represent continuous spectral content
    \item \textbf{Adaptive Resonator Model}: Learn or measure actual resonator frequency response for each instrument
    \item \textbf{Time-Varying Parameters}: Allow $\alpha_n$ and $\theta$ to vary with time, especially during attack phase
    \item \textbf{Better Fundamental Frequency Estimation}: Use more robust pitch tracking methods (autocorrelation, cepstral analysis)
    \item \textbf{Spectral Envelope Modeling}: Model overall spectral shape in addition to discrete harmonics
    \item \textbf{Real-Time Optimization}: Optimize the synthesis algorithm for low-latency real-time processing
\end{enumerate}

\subsection{Conclusion}

% TODO: Write conclusion
% - Mathematical model successfully captures primary harmonic structure and decay
% - Demonstrates feasibility of preserving acoustic sound quality in electric instruments
% - Extracted parameters ($A_n$, $\phi_n$, $f_0$, $\tau$, $A_0$) form compact representation
% - Can be applied to transform electric sasando signals
% - Combines electric input pitch with reference timbre
% - 400 harmonics ensure comprehensive frequency coverage
% - Dual-series approach effectively models resonator interactions
% - While perfect replication not achievable with purely harmonic model, captures essential characteristics
% - Future work on inharmonic modeling and sophisticated resonator representations can improve accuracy
% - Enables real-time implementation in electric sasando systems

% ============================================================================
% REFERENCES
% ============================================================================
\begin{thebibliography}{99}

\bibitem{oppenheim2010}
Oppenheim, A. V., \& Schafer, R. W. (2010). \textit{Discrete-Time Signal Processing} (3rd ed.). Prentice Hall. Section 2.6: Sinusoidal Signals and Phasors.

\bibitem{proakis2007}
Proakis, J. G., \& Manolakis, D. G. (2007). \textit{Digital Signal Processing} (4th ed.). Prentice Hall. Chapter 2: Discrete-Time Signals and Systems.

\bibitem{roads1996}
Roads, C. (1996). \textit{The Computer Music Tutorial}. MIT Press. Chapters on additive synthesis and spectral analysis.

\bibitem{smith2011}
Smith, J. O. (2011). \textit{Spectral Audio Signal Processing}. W3K Publishing. Available online: \url{https://ccrma.stanford.edu/~jos/sasp/}

\bibitem{fletcher1998}
Fletcher, N. H., \& Rossing, T. D. (1998). \textit{The Physics of Musical Instruments} (2nd ed.). Springer-Verlag. Chapters on string instruments and resonators.

\bibitem{librosa2020}
McFee, B., et al. (2020). librosa: Audio and Music Analysis in Python. \textit{Journal of Open Source Software}, 5(50), 2154. \url{https://doi.org/10.21105/joss.02154}

\bibitem{numpy2020}
Harris, C. R., et al. (2020). Array programming with NumPy. \textit{Nature}, 585, 357–362. \url{https://doi.org/10.1038/s41586-020-2649-2}

\bibitem{scipy2020}
Virtanen, P., et al. (2020). SciPy 1.0: fundamental algorithms for scientific computing in Python. \textit{Nature Methods}, 17, 261–272. \url{https://doi.org/10.1038/s41592-019-0686-2}

\end{thebibliography}

% ============================================================================
% APPENDIX
% ============================================================================
\appendix

\section{Mathematical Model Implementation}

\subsection{Core Functions}

The following Python functions implement the mathematical model described in Section 2.

\subsubsection{Signal Loading}
The \texttt{load\_audio()} function loads audio files and returns time, amplitude, and sample rate arrays.

\subsubsection{Envelope Extraction}
The \texttt{extract\_envelope()} function extracts amplitude envelope using Hilbert transform with optional smoothing.

\subsubsection{Fourier Series Reconstruction}
The \texttt{process\_fourier\_series()} function reconstructs signals using the dual Fourier series model (Equation~\ref{eq:main_model}).

\subsubsection{A$_n$ Extraction}
The \texttt{extract\_A\_n()} function implements the A$_n$ extraction formula (Equation~\ref{eq:A_n_extraction}).

\subsubsection{Resonator Response Function}
The \texttt{alpha\_n\_func()} function implements the resonator response (Equation~\ref{eq:alpha_n}).

\subsubsection{Decay Envelope Application}
The \texttt{apply\_exponential\_decay()} function applies exponential decay envelope to signals.

\subsection{Configuration Parameters}

Key model parameters:
\begin{itemize}
    \item \texttt{NUM\_HARMONICS = 400}: Number of harmonics
    \item \texttt{THETA = $\pi$/4}: Phase shift (45 degrees)
    \item \texttt{INCLUDE\_RESONATOR = True}: Include resonator effect
    \item \texttt{SASANDO\_FREQ\_MIN = 98} Hz: Minimum resonator frequency
    \item \texttt{SASANDO\_FREQ\_MAX = 1047} Hz: Maximum resonator frequency
    \item \texttt{ENVELOPE\_METHOD = 'hilbert'}: Envelope extraction method
    \item \texttt{ENVELOPE\_SMOOTHING = 100}: Smoothing window size
    \item \texttt{APPLY\_DECAY = True}: Apply exponential decay
\end{itemize}

\subsection{Complete Processing Pipeline}

\begin{enumerate}
    \item Load audio file
    \item Extract envelope
    \item Estimate fundamental frequency from FFT
    \item Extract $A_n$ and $\phi_n$ for each harmonic
    \item Reconstruct signal using Fourier series
    \item Apply decay envelope
    \item Compare with original signal
\end{enumerate}

\end{document}

