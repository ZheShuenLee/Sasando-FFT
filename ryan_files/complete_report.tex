\documentclass[12pt]{article}

\usepackage{hyperref}
\usepackage{listings} 
% Essential packages for formatting
\usepackage[utf8]{inputenc}
\usepackage{amsmath, amssymb}
\usepackage{graphicx}
\usepackage{multicol} % For optional two-column formatting
\usepackage{geometry}
\geometry{margin=1in}
\usepackage{titlesec}
\usepackage{appendix}
\usepackage{xcolor}
\usepackage{subcaption}
\usepackage{caption}
\usepackage{amsfonts}
\usepackage[justification=centering]{caption}
\usepackage{soul}
\usepackage{booktabs} % For better tables

\captionsetup{labelformat=simple, labelsep=period, labelfont=bf}

% Custom settings
\setlength{\parindent}{0pt} % Optional: remove indentation for paragraphs
\setlength{\parskip}{0.4em} % Optional: add spacing between paragraphs

\title{\LARGE Modification of an Electric Sasando\\[1ex] \large Problem A}

\author{Team 717}

\date{} %For no date

%%%% Document begins here

\begin{document}

% Full-width title page content
\maketitle

\begin{abstract}

In this paper, we attempt to modify the sound of an electric sasando in order to replicate the sound of its traditional counterpart. The sound generated by the instrument is treated as a signal, modulated by the lontar leaf resonator, which the electric sasando lacks. This modulation is represented mathematically as a dual Fourier series with frequency-dependent amplification and exponential decay. A Fast Fourier Transform (FFT) analysis confirmed that the electronics also introduce high-frequency noise and distortion. The proposed solution is a two-stage system. First, a low-pass filter with a 1047 Hz cutoff frequency is used. This key value mitigates the electronic noise and filters out sounds the natural resonator would not amplify. Second, a Digital Signal Processor (DSP) generates the calculated modulating signal, based on the Fourier model, and adds it to the electric sasando's signal. This restores the traditional timbre. The conclusion is that this combined system (a 1047 Hz filter and a DSP) can successfully replicate the authentic sound while retaining the electric version's portability.

\end{abstract}

\newpage

\tableofcontents

\newpage

\section{Introduction}

\subsection{Timbre}

Timbre refers to a sound's quality, independent of pitch, intensity, and loudness \cite{zhang2024}. The fundamental frequency and overtones of a sound determine its timbre, and these are affected by factors such as the characteristics of the materials and structure of the instrument producing the sound \cite{olaoye2021}. As a result, each musical instrument will have a signature timbre.

\subsection{Sasando}

Sasando is a traditional instrument from Rote Island, East Nusa Tenggara, Indonesia. It is made of a bamboo tube and a hand-formed, dried lontar leaf resonator, as shown in Figure~\ref{fig:1}(a). It produces sound by vibrating the strings attached to the bamboo tube, while the leaf resonator acts as a natural amplifier and signal modulator that gives the sasando its unique timbre.

Previously, efforts to increase the sasando's volume used a microphone placed near the instrument, but the sound of the strings was not balanced between the melody, rhythm, and bass sections \cite{bakok2017}. Under these conditions, electric sasandos were introduced by removing the leaf resonator and replacing it with a transducer connected to an amplifier. The setup can be seen in Figure~\ref{fig:1}(b), which includes a three-legged pole. Although the resulting volume is much higher and clearer, and the electric sasando is more portable, the sound no longer resembles the original sasando due to the change in timbre, instead sounding more like an electric guitar.

\begin{figure}[h!]
    \centering
    \includegraphics[width=0.9\linewidth]{sasandos.png}
    \caption{ (a) A traditional gong sasando; (b) Four electric sasandos. Photos taken by Edon Sasando \cite{bakok2017}. }
    \label{fig:1}
\end{figure}

Musical instruments are a tangible cultural heritage. To preserve this form of heritage, it is natural to strive to preserve the unique sound that a traditional Sasando has. The aim of this project is to modify the electric sasando, such that it replicates the sound produced by the traditional version. For this, we attempt to devise a method to modulate the electrical signal as it passes through the components of the amplifier. In the following sections, we develop a mathematical model to quantify the timbre added by the lontar leaf resonator and discuss a method to electronically introduce it to the signal.

\section{Theory}

\subsection{Fourier series and Fourier transform}

Musical instruments produce notes by creating vibrations in a medium, such as a string. In the case of a string instrument like the sasando, the frequency of this vibration will depend on the length of the string. In addition to the fundamental frequency of the note, there will be several higher frequency vibrations produced – higher harmonics. The frequencies of these overtones are integer multiples of the fundamental tone, as shown in Figure~\ref{fig:2}.

\begin{figure}[h!]
    \centering
    \includegraphics[width=0.5\linewidth]{integer multiples.png}
    \caption{A wave of a fundamental frequency (red) and the overtones with frequencies of its integer multiples.}
    \label{fig:2}
\end{figure}

The overtones constructively interfere with the fundamental frequency, introducing a level of complexity to the waveform of the note. The amplitude of each overtone depends on its interaction with the instrument – the absorption and reflection of the sound waves will be affected by the shape of the instrument and the material it is made of. As such, the resulting combination of harmonics, or timbre, is unique to each instrument. This is why the same note played on a guitar and a trumpet sounds very different, despite having the same fundamental frequency. 

Since the timbre of musical instruments can be described as a combination of several sine waves, it can be mathematically expressed as a Fourier series shown in Equation 1:

\begin{equation}
\sum_{n} A_{n} \sin\!\left( \frac{n \pi f_{n} t}{T} \right)
\end{equation}

where $A_n$ is the amplitude of the $n$-th harmonic, $f_n$ is its frequency, $T$ is the period of the fundamental tone, and $n$ is an integer. To express this equation visually, the G3 note waveform has been dissected into its constituent frequency parts in Figure~\ref{fig:3}. 

\begin{figure}[h!]
    \centering
    \includegraphics[width=1.0\linewidth]{constituent waveforms.png}
    \caption{The waveform for G note (196 Hz) is the interference of higher-order frequency waveforms with the fundamental frequency.}
    \label{fig:3}
\end{figure}

\subsection{The amplifying mechanisms of traditional and electric sasandos}

The sasando consists of strings connected to a bamboo tube. Plucking the strings generates the notes in the instrument. This sound is then amplified to make it easier for audiences to hear. As discussed above, traditional and electric sasandos make use of different mechanisms for this amplification. To develop a modification method for the electric sasando, it is important to look into the mechanisms in detail.

The traditional sasando makes use of a lontar leaf resonator to amplify its sound, as shown in Figure~\ref{fig:4}. The resonator reflects the sound waves incident on it – its curved geometry will lead to the sound waves being focused at a point, where they will interfere, amplifying the sound. However, the resonator adds a level of distortion to the sound wave. As the waves reflect off its surface, they will be slightly dampened. The level of amplitude dampening is wavelength dependent and is characteristic of the material and shape of the resonator. This can be thought of as a modulation of the incident note's waveform.

\begin{figure}[h!]
    \centering
    \includegraphics[width=0.8\linewidth]{traditional sasando amplifier.png}
    \caption{The modulation of sound through a traditional sasando.}
    \label{fig:4}
\end{figure}

\newpage

On the other hand, modern sasandos perform this amplification electrically, shown in Figure~\ref{fig:5}. Here, a transducer is attached to the bottom of the instrument, which detects the acoustic signal and converts it into an electrical signal. The electrical signal is then transmitted to the amplifier, which increases the voltage of the signal and converts it back into a sound wave.

\begin{figure}[h!]
    \centering
    \includegraphics[width=0.8\linewidth]{electric sasando amplifier.png}
    \caption{The amplification of sound through an electric sasando.}
    \label{fig:5}
\end{figure}

Since this system no longer has the lontar leaf resonator, the resulting sound wave will not have the added timbre, reducing the complexity of the sound. Furthermore, the electronics themselves will add noise to the signal, further distorting the sound. As a result, the electric sasando will no longer sound like its traditional counterpart.

In order to make the electric sasando's sound resemble that of the acoustic instrument, two modifications must be made. First, the resonator's timbre must be added to the sound, and the noise from the electronics must be removed.

\subsection{Mathematical Model for Resonator Modulation}

The complete model for replicating traditional sasando acoustic characteristics is:

\begin{equation}
f(t) = A_0 e^{-t/\tau} \cdot \left[ \sum_{n=0}^{N} A_n \cos\left(n\omega_0 t + \phi_n\right) + \sum_{n=0}^{N} \alpha_n A_n \cos\left(n\omega_0 t + \phi_n + \theta\right) \right]
\end{equation}

where $N = 399$ (400 harmonics: $n = 0$ to $399$). 

\begin{figure}[h!]
    \centering
    \includegraphics[width=0.8\linewidth]{sasando_spherical_thing.jpeg}
    \caption{The modulation of sound through a traditional sasando. The incident wave from the strings interacts with the curved resonator, producing a modified reflected wave that contributes to the overall sound. This interaction is modeled by the dual Fourier series, where the first series represents the original signal and the second series represents the resonator's contribution.}
    \label{fig:resonator_interaction}
\end{figure}

The model consists of three main components:

\begin{enumerate}
    \item \textbf{Harmonic Decomposition}: Signal represented as sum of harmonics with amplitudes $A_n$ and phases $\phi_n$
    \item \textbf{Resonator Effect}: Second series adds frequency-dependent amplification ($\alpha_n$) and phase shift ($\theta$)
    \item \textbf{Temporal Decay}: Exponential envelope ($A_0 e^{-t/\tau}$) captures natural decay of plucked strings
\end{enumerate}

\subsubsection{Model Components}

\textbf{First Series:} $\sum_{n=0}^{N} A_n \cos(n\omega_0 t + \phi_n)$ represents the original signal decomposed into harmonics, where:
\begin{itemize}
    \item $A_n$: Amplitude coefficient for the $n$-th harmonic (extracted from input signal)
    \item $\phi_n$: Phase coefficient for the $n$-th harmonic (extracted from input signal)
    \item $\omega_0 = 2\pi f_0$: Fundamental angular frequency (rad/s)
    \item $n$: Harmonic index (0 = DC, 1 = fundamental, 2, 3, ..., 399)
\end{itemize}

\textbf{Second Series:} $\sum_{n=0}^{N} \alpha_n A_n \cos(n\omega_0 t + \phi_n + \theta)$ represents the resonator's contribution, where:
\begin{itemize}
    \item $\alpha_n$: Frequency-dependent amplitude scaling factor (models resonator response)
    \item $\theta$: Phase shift introduced by the resonator (constant, $\pi/4$ rad)
\end{itemize}

\textbf{Exponential Decay Envelope:} $E(t) = A_0 e^{-t/\tau}$, where:
\begin{itemize}
    \item $A_0$: Initial amplitude (extracted from the signal's peak envelope)
    \item $\tau$: Decay time constant (fitted from the signal's envelope using exponential regression)
\end{itemize}

The envelope is extracted using the Hilbert transform.

\subsubsection{Resonator Response Function}

The function $\alpha_n$ models the frequency-dependent amplification of the sasando resonator:

\begin{equation}
\alpha_n = \begin{cases}
    0.8 & \text{if } 98 \text{ Hz} \leq n \cdot f_0 \leq 1047 \text{ Hz} \\
    0.2 & \text{otherwise}
\end{cases}
\end{equation}

This reflects the resonator's natural amplification of frequencies within the sasando's characteristic range (98-1047 Hz), where frequencies in this range are amplified more strongly compared to frequencies outside this range.

\section{Method}

\subsection{Mathematical Model – Characterising Timbre}

\subsubsection{Parameter Extraction}

When extracting the amplitude coefficients $A_n$ from a recorded signal (which already includes resonator effects), we must account for the resonator's contribution. The FFT magnitude $M_n$ at frequency $n\omega_0$ represents the combined magnitude of both series terms.

Starting with the sum of two cosine terms:
\begin{equation}
s(t) = A_n \cos(n\omega_0 t + \phi_n) + \alpha_n A_n \cos(n\omega_0 t + \phi_n + \theta)
\end{equation}

Using the trigonometric identity $\cos(a + b) = \cos(a)\cos(b) - \sin(a)\sin(b)$, this can be written as a single sinusoid $R \cos(n\omega_0 t + \phi_n + \delta)$, where the magnitude $R$ is:

\begin{equation}
R = A_n \sqrt{1 + 2\alpha_n \cos(\theta) + \alpha_n^2}
\end{equation}

Since $M_n$ (from FFT) represents the magnitude $R$ of the combined signal, we solve for $A_n$:

\begin{equation}
A_n = \frac{M_n}{\sqrt{1 + 2\alpha_n \cos(\theta) + \alpha_n^2}}
\end{equation}

\subsubsection{Model Assumptions}

The mathematical model is based on the following assumptions:
\begin{enumerate}
    \item \textbf{Harmonic Structure}: The signal can be represented as a sum of harmonics (integer multiples of the fundamental frequency). Inharmonic partials are negligible or approximated by nearby harmonics.
    \item \textbf{Linear Resonator Model}: The resonator's effect is modeled as a linear transformation with frequency-dependent amplitude scaling $\alpha_n$ and constant phase shift $\theta$. The resonator behaves linearly without significant non-linear distortion.
    \item \textbf{Exponential Decay}: The amplitude envelope follows an exponential decay model $A_0 e^{-t/\tau}$. The decay is uniform across all harmonics and the decay time constant $\tau$ is constant throughout the signal.
    \item \textbf{Stationary Signal}: The signal characteristics (fundamental frequency, harmonic amplitudes, phases) are stationary over the analysis window, except for the exponential decay envelope.
    \item \textbf{Perfect Harmonic Alignment}: All frequency components are exact integer multiples of the fundamental frequency $f_0$.
    \item \textbf{Noise-Free Extraction}: Noise and continuous spectral content do not significantly affect the extraction of harmonic amplitudes and phases.
    \item \textbf{Resonator Frequency Response}: The resonator's frequency response is approximated by a step function as shown in Equation (3).
    \item \textbf{Constant Phase Shift}: The phase shift $\theta$ introduced by the resonator is constant across all frequencies.
\end{enumerate}

\subsection{Feature Extraction}

In order to apply the mathematical model to recorded signals, we must extract the necessary parameters from the audio data. The feature extraction process involves several steps.

\subsubsection{Fundamental Frequency Estimation}

The fundamental frequency $f_0$ was estimated by identifying the peak magnitude in the FFT spectrum within the range 20-2000 Hz. This provides $\omega_0 = 2\pi f_0$ for the Fourier series model. The peak detection method ensures that the strongest frequency component is identified as the fundamental, which is appropriate for musical notes where the fundamental typically has the highest amplitude.

\subsubsection{Harmonic Detection and Extraction}

For each signal, 400 harmonics were extracted ($n = 0$ to $399$), ensuring comprehensive coverage of the frequency spectrum. The harmonic frequencies are calculated as $n \cdot f_0$ for each harmonic index $n$. This fixed number of harmonics was chosen to provide sufficient frequency resolution while remaining computationally manageable. The FFT magnitude $M_n$ and phase $\phi_n$ were extracted at each harmonic frequency, and the base amplitude $A_n$ was calculated using Equation (6) to account for the resonator effect.

\subsubsection{Envelope Extraction and Decay Fitting}

The amplitude envelope was extracted using the Hilbert transform method, which provides an accurate representation of the signal's amplitude modulation over time. The envelope was then smoothed using a moving average filter (window size: 100 samples) to reduce noise. An exponential decay model $A_0 e^{-t/\tau}$ was fitted to the extracted envelope using linear regression on the logarithmic envelope values. This provides the initial amplitude $A_0$ (peak of the envelope) and the decay time constant $\tau$.

\subsection{Effect of Electronics}

In order to account for the effect of the electronics, we must first investigate how an electric amplifier distorts the signal from a musical instrument. For this, the sound waves produced by an electric, acoustic, and classical guitar were compared. Guitars were used in this experiment since, like sasandos, they are string instruments and have their electric counterparts.

Recordings were obtained of the guitars playing the notes C (130.81Hz), D (146.83Hz), E (164.81Hz), F (174.61Hz), and G (196.00Hz) \cite{mixbutton2025}. Five recordings were taken of each note for all three instruments in the form of M4A files with the same sampling rate (48000 Hz per second); during data analysis, an average of the repeats was obtained to mitigate the effects of random error. The files were truncated to have the same length as the file with the shortest length. A fast Fourier transform (FFT) was performed using the Python NumPy efficient FFT algorithm to obtain the frequencies and their amplitudes that compose the wavefunction of each note. The amplitudes of the frequencies are normalized to show relative amplitude strengths compared to the fundamental frequency (in cases where the fundamental frequency has the maximum strength).  

As seen in the plots in Figure~\ref{fig:6}, the spectrum of the electric guitar shows prominent features at higher frequencies that are absent in the other two guitars. This may be due to the distortion introduced by the guitar's electric amplification system at higher frequencies. If so, electric sasandos would likely experience a similar effect. Therefore, the high-frequency overtones must be damped. 

\begin{figure}[h!]
    \centering
    \includegraphics[width=1.0\linewidth]{Comparison between instruments.png}
    \caption{Comparison of G3 note from electric, acoustic and classical guitars. The plots on the left show the waveform of the note; the plots on the right show the frequency amplitude distributions after Fourier transforms have been performed on the notes.}
    \label{fig:6}
\end{figure}

\section{Solution}

The challenge presented in this exercise was to ensure that the electric sasando sounds similar to its traditional counterpart. The sound signal produced by the two instruments was broken down into components, each representing the signal modulation caused by different parts of the instruments. Since the electric sasando's signal lacks the modulation created by the resonator and has an additional distortion caused by the electronics, the aim of our approach is to manipulate the signal to account for these differences. This can be accomplished by introducing two additional electronic components to the setup: a low-pass filter and a digital signal processor (DSP), as shown in Figure~\ref{fig:7}.

Firstly, the signal produced by the strings will be converted to an electric signal by the transducer. This signal can then be passed through a low-pass filter. The filter allows signals of frequencies below a threshold to pass through, while attenuating the components of the signal above this threshold. In order to filter out the high-frequency noise in the system, an appropriate threshold must be selected; a balance must be struck between mitigating the electronic distortion and including sufficient overtones to preserve the complexity of the instrument's sound. In the traditional instrument, the lontar leaf resonator amplifies frequencies up to 1047 Hz. The filter's threshold can thus safely be set to this value, as it best replicates this effect.

\begin{figure}[h!]
    \centering
    \includegraphics[width=0.9\linewidth]{2025 Question A (1).png}
    \caption{The modification of an electric sasando by adding a low-pass filter and a DSP.}
    \label{fig:7}
\end{figure}

After passing through the low-pass filter \cite{allaboutcircuits}, the signal must be sent through a DSP. DSPs \cite{utexas_dsp} are complex electronic components capable of analysing and modifying signals. These are programmable units. They are capable of generating sine waves of defined frequencies and using additive synthesis to combine the waves with the input signal. In fact, the sasando's electronics are already designed to facilitate this – it has a sound effect tool built-in, that allows it to produce sounds similar to other instruments such as the flute, organ, etc \cite{bakok2017}.

Since the electric sasando is already equipped with the circuitry to add modulation to the signal, the challenge lies in identifying the frequencies and amplitudes of the sine waves which need to be generated. These parameters will differ based on the fundamental frequency (i.e. note being played). Therefore, our mathematical model must be applied to all notes the sasando produces in order to extract the parameters and define a modulating function for each note. The modulating functions can be stored in the memory of the DSP itself. The processor can then perform a FFT to identify the fundamental frequency being played, and add the corresponding modulating signal to it.

Several practical considerations must be made here. Firstly, the DSP must be small and portable – the resonator was removed to enhance portability in the first place. While traditional synthesisers are large enough to impact the portability of the system, the DSP itself can be a microprocessor. The suitability of various microprocessors for signal processing applications has already been investigated in previous studies – \cite{campos2020} determined that RaspberryPi was the most effective option.

\section{Results}

The mathematical model was tested on recordings of acoustic, classical, and electric guitars playing the notes C, D, and E. For each instrument-note combination, the model was used to reconstruct the signal, and the reconstructed signal was compared to the original using both time-domain and frequency-domain metrics. Figure~\ref{fig:classical_results} shows an example comparison for classical guitar, demonstrating the model's ability to replicate both the time-domain waveform and frequency-domain spectrum.

\begin{figure}[h!]
    \centering
    \includegraphics[width=0.9\linewidth]{results_final/Classical guitar.png}
    \caption{Comparison of original and Fourier model signals for classical guitar. The plots show time-domain waveforms (top) and frequency-domain spectra (bottom) for notes C, D, and E.}
    \label{fig:classical_results}
\end{figure}

\subsection{Harmonic Analysis}

The model employed 400 harmonics ($n = 0$ to $399$) for all notes, ensuring comprehensive coverage of the frequency spectrum. The fundamental frequencies were accurately identified across all test cases, ranging from 129.89 Hz (electric guitar C) to 329.29 Hz (electric guitar E). Table~\ref{tab:summary} presents the fundamental frequencies, amplitude ratios, and decay parameters for each instrument-note combination.

\begin{table}[h!]
\centering
\caption{Summary of Fourier Model Matching Results}
\label{tab:summary}
\begin{tabular}{lcccc}
\toprule
Instrument & Note & Fundamental (Hz) & Max Ratio & RMS Ratio \\
\midrule
Acoustic Guitar & C & 262.63 & 0.04 & 0.34 \\
Acoustic Guitar & D & 145.90 & 0.08 & 0.39 \\
Acoustic Guitar & E & 163.96 & 0.01 & 0.04 \\
Classical Guitar & C & 131.14 & 0.01 & 0.06 \\
Classical Guitar & D & 293.85 & 0.01 & 0.04 \\
Classical Guitar & E & 164.75 & 0.01 & 0.06 \\
Electric Guitar & C & 129.89 & 0.01 & 0.08 \\
Electric Guitar & D & 146.26 & 0.03 & 0.32 \\
Electric Guitar & E & 329.29 & 0.01 & 0.16 \\
\bottomrule
\end{tabular}
\end{table}

The Max Ratio and RMS Ratio indicate how well the Fourier model matches the original signal's amplitude. A ratio close to 1.0 indicates good matching, while ratios less than 1.0 indicate that the model produces a quieter signal than the original. The results show that the model consistently produces signals with lower amplitudes than the originals, with RMS ratios ranging from 0.04 to 0.39. This suggests that while the model captures the harmonic structure and frequency content, the overall energy level is reduced. The acoustic guitar shows the highest RMS ratios (0.34 and 0.39 for notes C and D), while classical and electric guitars generally show lower ratios (0.01 to 0.08 for most notes).

\subsection{Decay Analysis}

The decay time constant $\tau$ was extracted from the signal envelope for each recording. Most signals showed infinite decay constants ($\tau = \infty$), indicating that the envelope did not exhibit clear exponential decay within the analysis window. This may occur when the signal duration is shorter than the decay time constant, or when the signal maintains relatively constant amplitude. Only two signals showed finite decay constants: classical guitar D ($\tau = 3.110$ s) and classical guitar E ($\tau = 10.546$ s). These longer decay times suggest that classical guitar notes sustain longer than acoustic or electric guitar notes, which is consistent with the instrument's characteristics.

\subsection{Time Domain Matching}

The time-domain comparison evaluates how well the reconstructed waveform matches the original signal's temporal characteristics. Table~\ref{tab:time} presents the correlation coefficients, root mean squared error (RMSE), and mean absolute error (MAE) for each instrument-note combination in the time domain.

\begin{table}[h!]
\centering
\caption{Time Domain Distribution Comparison Metrics}
\label{tab:time}
\small
\begin{tabular}{lcccc}
\toprule
Instrument & Note & Correlation & RMSE & MAE \\
\midrule
Acoustic Guitar & C & 0.2741 & 0.0075 & 0.0049 \\
Acoustic Guitar & D & 0.3936 & 0.0091 & 0.0055 \\
Acoustic Guitar & E & 0.4748 & 0.0091 & 0.0050 \\
Classical Guitar & C & 0.4378 & 0.0400 & 0.0199 \\
Classical Guitar & D & 0.4579 & 0.0415 & 0.0147 \\
Classical Guitar & E & 0.4759 & 0.0413 & 0.0175 \\
Electric Guitar & C & 0.4628 & 0.0083 & 0.0045 \\
Electric Guitar & D & 0.3558 & 0.0047 & 0.0025 \\
Electric Guitar & E & 0.1674 & 0.0041 & 0.0015 \\
\bottomrule
\end{tabular}
\end{table}

The correlation coefficients range from 0.1674 to 0.4759, indicating moderate positive correlations between the original and reconstructed signals. The classical guitar shows the highest correlations, particularly for notes D (0.4579) and E (0.4759), suggesting that the model performs best on classical guitar signals. The RMSE and MAE values are relatively small, with classical guitar notes showing slightly higher errors due to their larger amplitude ranges. The electric guitar E note shows the lowest correlation (0.1674), which may be due to its higher fundamental frequency (329.29 Hz) and the presence of more complex high-frequency content.

\subsection{Frequency Domain Matching}

The frequency-domain comparison evaluates how well the model replicates the harmonic structure and spectral content of the original signals. Table~\ref{tab:freq} presents the correlation coefficients, RMSE, and MAE for the frequency domain comparisons.

\begin{table}[h!]
\centering
\caption{Frequency Domain Distribution Comparison Metrics}
\label{tab:freq}
\small
\begin{tabular}{lcccc}
\toprule
Instrument & Note & Correlation & RMSE & MAE \\
\midrule
Acoustic Guitar & C & 0.2921 & 0.0078 & 0.0006 \\
Acoustic Guitar & D & 0.4193 & 0.0052 & 0.0004 \\
Acoustic Guitar & E & 0.5057 & 0.0040 & 0.0003 \\
Classical Guitar & C & 0.4658 & 0.0050 & 0.0002 \\
Classical Guitar & D & 0.7419 & 0.0053 & 0.0002 \\
Classical Guitar & E & 0.6044 & 0.0043 & 0.0002 \\
Electric Guitar & C & 0.4926 & 0.0038 & 0.0002 \\
Electric Guitar & D & 0.3818 & 0.0067 & 0.0010 \\
Electric Guitar & E & 0.1792 & 0.0133 & 0.0018 \\
\bottomrule
\end{tabular}
\end{table}

The frequency domain correlations are generally higher than the time domain correlations, with classical guitar D note achieving the highest correlation of 0.7419. This indicates that the model more accurately captures the harmonic structure and frequency content than the temporal waveform details. The classical guitar consistently shows the best frequency domain matching, with correlations above 0.46 for all three notes. The acoustic guitar E note also shows good frequency domain matching (0.5057), while the electric guitar E note shows poor correlation (0.1792), consistent with the time domain results.

\subsection{Quantitative Summary}

Across all test cases, the average time-domain correlation is 0.388, and the average frequency-domain correlation is 0.451. This indicates that the model performs better at replicating frequency content than temporal waveforms. The classical guitar shows the best overall performance, with average correlations of 0.457 (time domain) and 0.604 (frequency domain). The electric guitar shows the most variable performance, with correlations ranging from 0.167 to 0.463 in the time domain and 0.179 to 0.493 in the frequency domain.

\subsection{Instrument-Specific Observations}

The results reveal distinct characteristics for each instrument type:

\textbf{Classical Guitar:} The classical guitar demonstrates the best overall matching, with the highest frequency domain correlations (0.4658 to 0.7419) and strong time domain correlations (0.4378 to 0.4759). This instrument's clear harmonic structure and strong fundamental frequencies make it well-suited for the Fourier series model. The classical guitar also shows measurable decay constants for notes D and E, indicating longer sustain characteristics.

\textbf{Acoustic Guitar:} The acoustic guitar shows moderate matching, with frequency domain correlations ranging from 0.2921 to 0.5057. The note E achieves the best frequency domain correlation (0.5057) among acoustic guitar notes, while note C shows the lowest (0.2921). The more complex spectral content of acoustic guitars, particularly in the higher frequency ranges, presents challenges for the model. However, the acoustic guitar shows the highest amplitude ratios (RMS ratios of 0.34 and 0.39 for notes C and D), suggesting better energy matching.

\textbf{Electric Guitar:} The electric guitar exhibits variable performance, with note C showing good frequency domain correlation (0.4926) but note E showing poor correlation (0.1792). The electric guitar's additional high-frequency content from electronic amplification may contribute to the discrepancies observed, particularly for higher-pitched notes. The time domain correlation for electric guitar E (0.1674) is the lowest among all test cases, indicating significant challenges in modeling this particular combination.

\section{Discussion}

\subsection{Strengths of our method}

\begin{itemize}
    \item Portability of the system is retained.
    \item Easy to implement.
    \item The mathematical model provides a quantitative framework for characterizing the resonator's effect on timbre.
    \item The dual Fourier series approach captures both harmonic structure and frequency-dependent amplification.
    \item Exponential decay modeling accurately represents the natural decay characteristics of plucked strings.
    \item The model successfully identifies fundamental frequencies across a wide range (129.89 Hz to 329.29 Hz).
    \item Frequency domain matching shows strong correlations, particularly for classical guitar (up to 0.7419).
    \item The feature extraction process reliably extracts harmonic amplitudes and phases from recorded signals.
\end{itemize}

\subsection{Weaknesses of our method}

We simplified the complex system to a series of components in both cases. We use a signal modulator to add in the timbre, and a low-pass filter to mitigate distortion.

\textbf{Limitations:}

Comparing the electric guitar to the classical and acoustic guitars makes it difficult to conclude whether the differences are purely due to the electronic amplifier, as the instruments also use different materials. However, since the lontar leaf resonator naturally amplifies frequencies only up to about 1047 Hz, any additional high-frequency amplification from the electronic system can be controlled. A low-pass filter will ensure that the resulting sound still resembles the acoustic instrument.

The microcontroller used to generate the timbre is a small component and cannot produce a large number of harmonics, so it will be limited in how many frequencies it can add. By selecting only the most prominent harmonics, the resulting sound can be made as close as possible to the original.

Additional noise sources will be introduced by the electronic system, and it is not practical to account for or eliminate all of them.

The mathematical model makes several assumptions that may not hold perfectly in practice:
\begin{itemize}
    \item The assumption of perfect harmonic alignment may not account for slight frequency variations in real instruments.
    \item The step function approximation of the resonator frequency response is a simplification of the actual complex frequency response.
    \item The constant phase shift assumption may not hold across all frequencies in reality.
    \item Noise and continuous spectral content may affect parameter extraction to some degree.
    \item The model produces signals with lower amplitudes than the originals (Max Ratio and RMS Ratio less than 1.0), indicating that energy matching needs improvement. This may require additional gain adjustment in the DSP implementation.
    \item Time domain correlations are moderate (0.17 to 0.48), suggesting that temporal waveform details are not fully captured. This may be due to transient effects during the attack phase that are not modeled.
    \item Some notes, particularly electric guitar E (frequency correlation 0.1792), show poor matching, indicating limitations for certain frequency ranges or instrument types. The higher fundamental frequency (329.29 Hz) and complex high-frequency content may be contributing factors.
    \item Most signals show infinite decay constants, suggesting that the exponential decay model may not be appropriate for all signal types, or that the analysis window is too short to capture decay characteristics.
\end{itemize}

\subsection{Practical Implementation Considerations}

For the DSP implementation, several practical considerations must be addressed. The computational requirements for processing 400 harmonics in real-time may be significant, particularly for a portable microprocessor. However, since the results show that the model performs well with fewer prominent harmonics (as evidenced by the good frequency domain correlations), it may be possible to reduce the number of harmonics processed in real-time without significant quality loss. The DSP must be capable of performing FFT analysis to identify the fundamental frequency, and then apply the appropriate modulating function from stored memory. The low amplitude ratios observed in the results suggest that gain adjustment will be necessary in the final implementation to match the original signal levels.

\section{Conclusion}

The mathematical model developed in this work provides a framework for replicating the acoustic characteristics of traditional sasando instruments in electric versions. The dual Fourier series representation, incorporating frequency-dependent amplification and exponential decay, successfully captures the essential timbre components of string instruments.

The results demonstrate that the model performs best on classical guitar signals, achieving frequency domain correlations up to 0.7419. This suggests that instruments with clear harmonic structures and strong fundamental frequencies are well-suited for this approach. The model's ability to accurately identify fundamental frequencies across a wide range (129.89 Hz to 329.29 Hz) confirms its robustness for different musical notes.

However, the results also reveal limitations. The model consistently produces signals with lower amplitudes than the originals, and time domain correlations remain moderate. These discrepancies can be attributed to the model's assumptions, including perfect harmonic alignment, simplified resonator response, and the inability to represent continuous spectral content or inharmonic partials. The poor performance on electric guitar E note (correlation 0.1792) indicates that the model may have difficulty with higher-frequency notes or instruments with complex electronic amplification characteristics.

For practical implementation, the two-stage system proposed – a low-pass filter with 1047 Hz cutoff frequency combined with a DSP – provides a viable solution. The filter mitigates high-frequency electronic noise while preserving the frequency range that the traditional resonator amplifies. The DSP can then apply the mathematical model to add the resonator's characteristic timbre to the electric signal. The computational requirements can be managed by selecting the most prominent harmonics for real-time processing, and gain adjustment can address the amplitude matching issues.

Future work could focus on improving amplitude matching, refining the resonator response function to better match actual acoustic behavior, and developing adaptive parameter extraction methods that account for instrument-specific characteristics. Despite the limitations, this approach demonstrates that electronic modification of electric sasandos can successfully replicate key aspects of traditional acoustic timbre while maintaining the portability advantages of electric instruments.

\section{References}

\begin{thebibliography}{9}

\bibitem{zhang2024}
Zhang, H., Lin, J., \& Chen, S. (2024). Timbre Perception, Representation, and its Neuroscientific Exploration: A Comprehensive Review. \textit{ArXiv}. https://arxiv.org/abs/2405.13661 

\bibitem{olaoye2021}
Olaoye, K. O. (2021). Timbre harmonic model for measuring sound harmony: A case study of Gmelina arborea wood. \textit{Applied Acoustics, 177}, 107925. 

\bibitem{bakok2017}
Bakok, Y. (2017). Electric Sasando of East Nusa Tenggara, Indonesia. \textit{International Journal of Creative and Arts Studies}, 1(2), 84. 

\bibitem{mixbutton2025}
Daniel. (2025, January 27). \textit{Music note to frequency chart}. MixButton. \newline \href{https://mixbutton.com/music-tools/frequency-and-pitch/music-note-to-frequency-chart}{https://mixbutton.com/music-tools/frequency-and-pitch/music-note-to-frequency-chart} 

\bibitem{allaboutcircuits}
Nordic Semiconductor. (n.d.). \textit{Low-pass filters.} All About Circuits. \newline \href{https://www.allaboutcircuits.com/textbook/alternating-current/chpt-8/low-pass-filters/}{https://www.allaboutcircuits.com/textbook/alternating-current/chpt-8/low-pass-filters/}

\bibitem{utexas_dsp}
\textit{Memory and DSP Processors: A Brief History and Survey of On-Going Innovations}. University of Texas at Austin. \newline \href{https://users.ece.utexas.edu/~bevans/courses/ee382c/lectures/02_signal_processing/project1.html}{https://users.ece.utexas.edu/\~bevans/courses/ee382c/lectures/02\_signal\_processing/project1.html} 

\bibitem{campos2020}
Campos, J. D., \& Fonseca, W. D. (2020). Portable digital audio synthesizer assembly with open-source software. \textit{Proceedings of the Institute of Acoustics}, 42(3). 

\bibitem{oppenheim2010}
Oppenheim, A. V., \& Schafer, R. W. (2010). \textit{Discrete-Time Signal Processing} (3rd ed.). Prentice Hall.

\bibitem{proakis2007}
Proakis, J. G., \& Manolakis, D. G. (2007). \textit{Digital Signal Processing} (4th ed.). Prentice Hall.

\end{thebibliography}

\section{Appendix}

\subsection{Python script for Fourier transforms}

\lstinputlisting{Fourier transform (4).py}

\end{document}

